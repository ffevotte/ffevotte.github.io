\documentclass{moderncv}
\usepackage[a4paper,hmargin=1.5cm,vmargin={1.5cm, 2cm}]{geometry}
\usepackage[utf8]{inputenc}
\moderncvstyle{ff}
%\moderncvstyle{classic}
\moderncvcolor{blue}
\usepackage{XCharter}
\usepackage{fixltx2e}
\usepackage{totcount}

\setlength{\hintscolumnwidth}{2.8cm} % width of the column with the dates

\nopagenumbers{} % to suppress automatic page numbering for CVs longer than one page

\usepackage{enumitem}
\setlist[description]{font=\normalfont\itshape}
\setlist[itemize]{noitemsep,nosep}

\name{François}{Févotte}
\title{Mathématiques appliquées et \mbox{calcul scientifique à haut rendement}}
\address{TriScale Innov}{7 rue de la Croix Martre}{91120 Palaiseau}
%\phone[mobile]{+1~(234)~567~890}
%\phone[fixed]{(+33)~1~78~19~44~23}
%\phone[fax]{+3~(456)~789~012}
\email{ff@triscale-innov.fr}
%\homepage{www.johndoe.com}
\social[linkedin]{ffevotte}
%\social[twitter]{jdoe}
\social[github]{ffevotte}
%\extrainfo{additional information}
%\photo[64pt][0.4pt]{picture}
%\quote{Some quote}


\newcommand{\verrou}{\href{http://github.com/edf-hpc/verrou}{\textsc{Verrou}}}

\begin{document}
\makecvtitle
\newtotcounter{NbRefs}
\newcounter{BibIndex}
\setcounter{BibIndex}{\totvalue{NbRefs}}

\section{Expérience Professionnelle}

\cventry{depuis 2019}{Co-fondateur \& CTO}{TriScale Innov}{}{}{}

\cventry{depuis 2008}{Ingénieur Chercheur}{EDF R\&D, dept PERICLES, groupe Analyse et
  Modèles Numériques}{}{}{
  \begin{description}
  \item[Future Chaîne de Calcul des C{\oe}urs]: développement de solveurs pour
    la plate-forme de simulation neutronique \textsc{Cocagne}, membre du comité de
    suivi d'\textsc{Apollo3} dans l'Institut Tri-Partite EDF--CEA--AREVA (I3P) ;
  \item[Performance et Qualité des Simulations]: reponsable du lot ``sujets
    exploratoires et méthodes avancées'', développement de l'outil \verrou{}
    pour la qualité numérique des outils de calcul ;
  \item[Écoles d'été CEA--EDF--Inria]: membre du comité exécutif, représentant
    EDF pour l'analyse numérique.
  \end{description}
}

\cventry{depuis 2007}{Enseignement}{ENSTA {\em et} Université Paris Saclay}{}{}{%
  \begin{itemize}
  \item \textit{Introduction à la discrétisation des EDP} :
    1\textsuperscript{ère}~année~ENSTA, chargé de TD (resp: P. Joly) ;
  \item \textit{La méthode des éléments finis} :
    2\textsuperscript{ème}~année~ENSTA, chargé de TD (resp: S. Fliss) ;
  \item \textit{Modélisation et simulation du transport de particules neutres} :
    Master Analyse, Modélisation et \mbox{Simulation} (AMS), co-responsable avec
    G. Samba (CEA) ;
  \item Membre du conseil de perfectionnement de l'université Paris Saclay.
  \end{itemize}
}

\cventry{2005\\\footnotesize(6 mois)}{Stage de fin d'études}{AREVA NP,
  département Développement}{}{}{%
  \begin{itemize}
  \item Développement d'un modèle de sections efficaces pour la simulation
    neutronique des c{\oe}urs de réacteurs nucléaires à haute température
    (GTMHR).
  \item Utilisation des codes de calcul du CEA : \textsc{Apollo}2,
    \textsc{Cronos}2
  \end{itemize}
}

\cventry{2004\\\footnotesize(4 mois)}{Stage de recherche}{Univ. de Californie, Santa Barbara
  (UCSB), dept. Génie Chimique}{}{}{%
  \begin{itemize}
  \item Analyse de sensibilité d'un processus de granulation
  \item Développement d'une méthode de contrôle de la distribution de tailles de
    granulats produits.
  \end{itemize}
}

\section{Formation}

\cventry{2005--2008}{Thèse}{Commissariat à l'Énergie Atomique}{Saclay}{}{
  \textit{Sujet} : Simulation du transport des neutrons avec la méthode des
  caractéristiques.\\
  \textit{Encadrement} : Richard Sanchez, Simone Santandrea
  \begin{itemize}
  \item Développements intégrés dans le solveur TDT du code \textsc{Apollo2}.
  \end{itemize}
}

\cventry{2004--2005}{Master Modélisation et Simulation}{INSTN}{Saclay}{}{}

\cventry{2002--2005}{Ingénieur}{École Nationale Supérieure de Techniques Avancées (ENSTA)}{}{}{
  \begin{description}
  \item[Spécialité ``modélisation des systèmes physiques'' :] calcul
    scientifique et parallèle, commande des systèmes, optimisation
    différentiable, physique des réacteurs, propagation des ondes.
  \end{description}}

\section{Prix et distinctions}

\cventry{2007}{\itshape Best student paper award}{conférence Mathematics \&
  Computations}{}{}{%
  Lauréat de l'un des 3 prix décernés par un jury de l'ANS (\textit{American
    Nuclear Society}) parmi tous les étudiants participant à la conférence.%
}

\cventry{2007}{\itshape Best student award}{International Youth Conference in
  Energetics (IYCE)}{}{}{%
  Prix de la meilleure présentation à la conférence IYCE, dans la catégorie
  ``Simulation''.%
}

\newpage

\section{Compétences et sujets de recherche}

\subsection{Communication et animation d'activités scientifiques}

\cvitem{Encadrement \& Enseignement}{\vspace{-0.85em}
  \begin{itemize}
  \item Suivi et co-direction de quatre thèses avec divers partenaires : École
    Polytechnique de Montréal, Imperial College, Université Pierre et Marie
    Curie, CEA ;
  \item Encadrement de 7 stagiaires ;
  \item Enseignement académique en mathématiques appliquées (niveau L3 à M2) ;
  \item Formation professionnelle : organisation et intervention dans plusieurs
    écoles d'été.
  \end{itemize}\vspace{-1em}\strut
}

\

\cvitem{Communication scientifique}{\theBibIndex\ articles et présentations dans
  des revues et conférences internationales, ainsi que \begin{itemize}
  \item Relecture d'articles pour les revues \textit{Nuclear Science and
      Engineering}, \textit{Annals of Nuclear Energy}, \textit{European Physical
    Journal in Nuclear Sciences \& Technologies} ;
  \item Membre du comité de programme des workshops \& conférences :
    \begin{itemize}
    \item \textit{Numerical Software Verification} (NSV 2017),
    \item \textit{Euro-Par} 2019,
    \item \textit{Numerical Reproducibility at Exascale} (NRE 2019) ;
    \end{itemize}
  \item \textit{Panelist} pour le workshop \textit{Numerical
      Reproduciblity at Exascale} 2019.
  \end{itemize}\vspace{-1em}\strut
}

\subsection{Développement de solveurs à haut rendement, Mathématiques
  appliquées}
\cvitem{Méthodes num.}{Méthode des caractéristiques, éléments finis ;}

\cvitem{Langages}{\vspace{-0.85em}\begin{itemize}
  \item[] \hspace{-1.2em}{\itshape(production)} C++, Julia, Python ;
  \item[] \hspace{-1.2em}{\itshape(prototypage)} Matlab, Maxima (calcul formel) ;
  \end{itemize}\vspace{-1em}\strut}

\cvitem{Paradigmes}{Méta-programmation, \textit{multithreading}, vectorisation.}

\subsection{Qualité numérique des codes}

\cvitem{Arithmétique}{Arithmétiques flottante et stochastique, détection des
  erreurs d'arrondi ;}

\cvitem{Outils d'analyse}{Analyse des instabilités numériques :
  \begin{itemize}
  \item Utilisation de la bibliothèque \textsc{Cadna} (LIP6) ;
  \item Développement de l'outil \verrou{} et application aux codes métier EDF
    (Code\_Aster, Apogene) ;
  \item Montage de l'ANR Interflop, regroupant 7 partenaires pour le
    développement d'outils innovants pour la qualité numérique.
  \end{itemize}\vspace{-1em}\strut
}

\subsection{Physique des Réacteurs : modélisation et simulation}
\cvitem{Chaîne industrielle}{%
  Développement dans la future chaîne de calcul des c{\oe}urs \textsc{Cocagne} :
  \begin{itemize}
  \item Solveurs d'évolution des concentrations isotopiques (modèle général,
    modèle spécifique aux chaînes Xénon et Samarium),
  \item Méthodes de reconstruction fine du flux pour la diffusion neutronique,
  \item Fonctionnalités de \textit{rod cusping} pour les calculs
    instationnaires en diffusion.
  \end{itemize}\vspace{-1em}\strut
}

\cvitem{Schémas avancés}{%
  Développement de solveurs pour l'équation de Boltzmann :
  \begin{itemize}
  \item Mise au point de la méthode d'accélération du solveur de transport
    3D cartésien \textsc{Domino},
  \item Développement du solveur de transport 3D en géométries non structurées
    \textsc{Micado}.
  \end{itemize}\vspace{-1em}\strut
}

\subsection{Gestion d'activités \& projets techniques}

\cvitem{Schémas avancés en neutronique}{%
  Responsable du lot ``Solveurs'' du projet (2017--2018) :
  \begin{itemize}
  \item Réalisation de la feuille de route de développement des solveurs de
    neutronique pour la période 2019--2023 (800k€ sur 5~ans) ;
  \item Coordination des plans de R\&D relatifs aux solveurs neutroniques avec
    Framatome.
  \end{itemize}\vspace{-1em}\strut
}

\cvitem{Performance et Qualité des Simulations}{%
  Responsable du lot ``Sujets Exploratoires et Méthodes Avancées'' (2015--2019) :
  \begin{itemize}
  \item Organisation et suivi des activités du lot (2 thèses), en relation avec
    les acteurs d'autres départements d'EDF.
  \end{itemize}\vspace{-1em}\strut
}

\newpage

\section{Développement logiciel}

\subsection{Logiciels \textit{Open Source} \normalsize\normalfont(liste complète
  disponible sur \texttt{http://github.com/ffevotte})}

\vspace{0.5em}

\newcommand{\project}[3]{\cvitem{#1\\[0.2em]
    \footnotesize\color{gray}
    #2%
  }{%
    #3
  }}

\newcommand{\OSproject}[5]{\project{#1}{%
    #2\\%
    \faStar~#3\quad\faCodeFork~#4}{%
    #5
  }}

\OSproject{Verrou}{C++ (9k lignes)}{15}{3}{%
  Outil de diagnostic \& déboguage des problèmes liés à l'arithmétique flottante
  dans les codes de calcul scientifique.
  \begin{itemize}
  \item Co-développeur principal
  \item Outil utilisé par de nombreux développeurs de codes : EDF, CEA, Thalès,
    CNRS/IN2P3\ldots
  \item \url{https://github.com/edf-hpc/verrou}
  \end{itemize}
}

\OSproject{clang-tags}{C++ (4k lignes)}{100}{21}{%
  Outil d'indexation de code source C / C++ pour les IDE.
  \begin{itemize}
  \item Développeur principal
  \item \url{https://github.com/ffevotte/clang-tags}
  \end{itemize}
}

\OSproject{StochasticArithmetic.jl}{Julia (1k lignes)}{5}{2}{\qquad%
  Paquet implémentant l'arithmétique stochastique en langage Julia.
  \begin{itemize}
  \item Développeur principal
  \item \url{https://github.com/ffevotte/StochasticArithmetic.jl}
  \end{itemize}
}

\OSproject{desktop-plus}{Lisp (1k lignes)}{43}{10}{%
  Extensions à la bibliothèque standard de gestion de sessions d'Emacs.
  \begin{itemize}
  \item Développeur principal
  \item \url{https://github.com/ffevotte/desktop-plus}
  \end{itemize}
}

\subsection{Logiciels propriétaires}

\project{Micado}{C++ (11k lignes)}{%
  Solveur pour l'équation de Boltzmann en géométries 3D prismatiques.
  \begin{itemize}
  \item Développeur principal : architecture, méthodes numériques et
    implémentation
  \item Méthode numérique à l'état de l'art pour les algorithmes de
    \textit{splitting} 2D--1D
  \item Parallélisme à 3 niveaux : MPI, tbb, SIMD
  \end{itemize}
}

\project{Domino}{C++ (100k lignes)}{%
  Solveur pour l'équation de Boltzmann en géométries cartésiennes, intégré dans
  la plate-forme de neutronique \textsc{Cocagne}.
  \begin{itemize}
  \item Contributeur : méthodes numériques pour l'accélération
  \item Mise au point du préconditionneur parallèle (mémoire distribuée)
  \end{itemize}
}

\project{Diabolo}{C++ (100k lignes)}{%
  Solveur pour l'équation de la diffusion en géométrie cartésiennes, intégré
  dans la plate-forme de neutronique \textsc{Cocagne}.
  \begin{itemize}
  \item Contributeur : solveur instationnaire, évolutions du schéma de
    discrétisation spatiale
  \item Mise au point des méthodes de remaillage lors de changement de géométrie
    au cours du temps
  \item Post-traitement des éléments finis pour améliorer l'ordre de convergence
    spatiale
  \end{itemize}
}

\section{Encadrement, enseignement \& formation}

\subsection{Enseignement académique}

\cventry{depuis 2007}{ENSTA et Université Paris Saclay}{50h/an}{}{}{%
  \begin{itemize}
  \item \textit{Introduction à la discrétisation des EDP} :
    1\textsuperscript{ère}~année~ENSTA, chargé de TD (resp: P. Joly) ;
  \item \textit{La méthode des éléments finis} :
    2\textsuperscript{ème}~année~ENSTA, chargé de TD (resp: S. Fliss) ;
  \item \textit{Modélisation et simulation du transport de particules neutres} :
    Master Analyse, Modélisation et \mbox{Simulation} (AMS), co-responsable avec
    G. Samba (CEA);
  \item Membre du conseil de perfectionnement de l'université Paris Saclay.
  \end{itemize}
}

\subsection{Formation professionnelle}

\cventry{2018}{Ecole d'été CEA--EDF--Inria}{40h}{}{}{%
  Organisation d'une école d'été d'une semaine sur le thème de la qualité
  numérique des codes :
  \begin{itemize}
  \item recrutement des formateurs ;
  \item préparation des séances de Travaux Pratiques autour du logiciel
    \textsc{Verrou} et des algorithmes compensés.
  \end{itemize}
}

\cventry{2018}{Ecole Thématique de la Simulation Numérique}{12h}{}{}{%
  Intervention lors d'une école co-organisée par le CEA/DAM et l'ENS Cachan
  (CMLA) :
  \begin{itemize}
  \item cours magistral sur l'arithmétique stochastique pour vérifier la qualité
    numérique des codes ;
  \item animation des Travaux Pratiques autour du logiciel \textsc{Verrou}.
  \end{itemize}
}

\cventry{2017}{Ecole d'été PRECIS {\footnotesize(Précision et REproductibilité
    en Calcul Informatique et Scientifique)}}{3h}{}{}{%
  Intervention lors d'une école d'été organisée par le groupe ``Calcul'' du CNRS
  :
  \begin{itemize}
  \item présentation de l'arithmétique stochastique et du logiciel
    \textit{Verrou},
  \item travaux pratiques.
  \end{itemize}
}

\subsection{Encadrement de doctorants}

\cventry{Wesley Ford\\(2016--)}{\bfseries Vers des méthodes d'accélération
  stables et efficaces en contextes parallèles appliquées à l'équation du
  transport des neutrons}{}{}{}{%
  \begin{itemize}
  \item Directeur : Christophe Calvin (CEA, Maison de la Simulation).
  \end{itemize}}

\cventry{Romain Picot\\(2015--2018)}{\bfseries Vérification numérique de
  logiciels de calculs industriels}{}{}{}{%
  \begin{itemize}
  \item Directrice : Fabienne Jézéquel (Université Pierre et Marie Curie, Paris 6).
  \end{itemize}}

\cventry{Rebecca Jeffers\\(2013--2016)}{\bfseries\itshape Goal Based Coupled
  Adaptive Mesh Refinement (AMR) and angular adaptivity on Cartesian Meshes for
  Modelling Neutron Transport in PWR Reactor Cores}{}{}{}{%
  \begin{itemize}
  \item Directeur : Matthew D. Eaton (Imperial College London).
  \end{itemize}}

\cventry{\hspace{-5em}Marc-André Lajoie\\(2011--2017)}{\bfseries Développement
  d’un schéma de calcul prismatique généralisé parallèle en transport
  déterministe hétérogène 3-D}{}{}{}{%
  \begin{itemize}
  \item Co-directeur : Guy Marleau (École Polytechnique de Montréal).
  \end{itemize}}

\subsection{Encadrement de stagiaires}

\cventry{Morgane Steins\\(2018)}{Amélioration de la convergence spatiale des
  éléments finis en neutronique}{}{}{}{%
  \begin{itemize}
  \item Utilisation de techniques d'analyse d'erreur \textit{a posteriori} pour
    améliorer la convergence spatiale des calculs en éléments finis de
    Raviart-Thomas ;
  \item Application à la neutronique : amélioration du calcul des intégrales du
    flux cellule par cellule pour la reconstruction fine de puissance ;
  \item Expérimentation numérique en langage Julia.
  \end{itemize}}

\cventry{Mehdi Ouafi\\(2017)}{Conditionnement des problèmes d'optimisation de la
  production hydraulique}{}{}{}{%
  \begin{itemize}
  \item Etude des liens entre conditionnement des matrices de contraintes et
    performance de la résolution des problèmes d'optimisation.
  \end{itemize}}

\cventry{Louis Denoix\\(2017)}{Stabilité numérique des algorithmes
  d'optimisation}{}{}{}{%
  \begin{itemize}
  \item Analyse de la qualité numérique du solveur d'optimisation SoPlex à
    l'aide de l'outil Verrou.
  \end{itemize}}

\cventry{Chou Xiaochen\\(2016)}{Gestion des erreurs numériques dans
  les problèmes de planification hydraulique}{}{}{}{%
  \begin{itemize}
  \item Analyse de l'impact de l'arithmétique flottante sur la faisabilité et la
    performance de résolution des problèmes de planification de la production
    hydraulique.
  \end{itemize}}

\cventry{Guy Augarde\\(2015)}{Exploration du potentiel du langage Julia pour le
  calcul scientifique}{}{}{}{%
  \begin{itemize}
  \item Développement d'une bibliothèque d'algèbre linéaire par blocs
    hiérarchiques en langage Julia.
  \end{itemize}}

\cventry{Salli Moustafa\\(2013)}{Vectorisation d'un algorithme de transport
  neutronique en géométries prismatiques}{}{}{}{%
  \begin{itemize}
  \item Amélioration des performances du solveur en utilisant les unités de
    calcul SSE / AVX,
  \item Utilisation de la bibliothèque C++ Eigen pour le développement de code
    vectorisé performant et portable.
  \end{itemize}}

\cventry{Ulrick Séverin\\(2012)}{Intégration fine du flux dans la plate-forme
  \textsc{Cocagne}}{}{}{}{%
  \begin{itemize}
  \item Mise au point et application d'une méthodologie de validation des
    techniques de reconstruction fine du flux dans la plate-forme de neutronique
    \textsc{Cocagne}.
  \end{itemize}}

\newpage

\section{Publications}

% \documentclass{moderncv}
\usepackage[a4paper,hmargin=1.5cm,vmargin={1.5cm, 2cm}]{geometry}
\usepackage[utf8]{inputenc}
\moderncvstyle{ff}
\moderncvcolor{blue}
\usepackage{totcount}
\usepackage{enumitem}

\name{}{}
\title{}

\begin{document}
\makecvtitle


\input{bibtype-\bibname}

\bibliographystyle{moderncv}
\newtotcounter{NbRefs}
\newcounter{BibIndex}
\setcounter{BibIndex}{\totvalue{NbRefs}}
\newcommand{\bibindex}{\theBibIndex\addtocounter{BibIndex}{-1}\stepcounter{NbRefs}}
\bibliography{../publications/biblio}

\end{document}

\bibliographystyle{moderncv}
\newcommand{\bibindex}{\theBibIndex\addtocounter{BibIndex}{-1}\stepcounter{NbRefs}}
% \bibliography{../publications/biblio}

\subsection{Journaux}\vspace{0.5em}
\input{biblio-article.bbl}

\subsection{Conférences}\vspace{0.5em}
\input{biblio-inproceedings.bbl}

\subsection{Thèse}\vspace{0.5em}
\input{biblio-phdthesis.bbl}

\subsection{\itshape Preprints}\vspace{0.5em}
\input{biblio-unpublished.bbl}

\end{document}
