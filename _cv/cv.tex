\documentclass{moderncv}
\usepackage[a4paper,hmargin=1.5cm,vmargin={1.1cm, 2cm}]{geometry}
\usepackage[utf8]{inputenc}
\moderncvstyle{ff}
%\moderncvstyle{classic}
\moderncvcolor{blue}
\usepackage{XCharter}
\usepackage{fixltx2e}
\usepackage{totcount}

\setlength{\hintscolumnwidth}{2.8cm} % width of the column with the dates

\nopagenumbers{} % to suppress automatic page numbering for CVs longer than one page

\usepackage{enumitem}
\setlist[description]{font=\normalfont\itshape}
\setlist[itemize]{noitemsep,nosep}

\name{François}{Févotte}
\title{Mathématiques appliquées et \mbox{calcul scientifique à haut rendement}}
\address{EDF Lab Paris Saclay}{7 bd Gaspard Monge}{91120 Palaiseau}
%\phone[mobile]{+1~(234)~567~890}
\phone[fixed]{(+33)~1~78~19~44~23}
%\phone[fax]{+3~(456)~789~012}
\email{francois.fevotte@edf.fr}
%\homepage{www.johndoe.com}
%\social[linkedin]{john.doe}
%\social[twitter]{jdoe}
\social[github]{ffevotte}
%\extrainfo{additional information}
%\photo[64pt][0.4pt]{picture}
%\quote{Some quote}


\newcommand{\verrou}{\href{http://github.com/edf-hpc/verrou}{\textsc{Verrou}}}

\begin{document}
\makecvtitle

\section{Expérience}

\cventry{depuis 2008}{Ingénieur Chercheur}{EDF R\&D, dept PERICLES, groupe Analyse et
  Modèles Numériques}{}{}{
  \begin{description}
  \item[Future Chaîne de Calcul des C{\oe}urs]: développement de solveurs pour
    la plate-forme de simulation neutronique \textsc{Cocagne}, membre du comité de
    suivi d'\textsc{Apollo3} dans l'Institut Tri-Partite EDF--CEA--AREVA (I3P) ;
  \item[Performance et Qualité des Simulations]: reponsable du lot ``sujets
    exploratoires et méthodes avancées'', développement de l'outil \verrou{}
    pour la qualité numérique des outils de calcul ;
  \item[Écoles d'été CEA--EDF--INRIA]: membre du comité exécutif, représentant
    EDF pour l'analyse numérique.
  \end{description}
}

\cventry{depuis 2007}{Enseignement}{ENSTA {\em et} Université Paris Saclay}{}{}{%
  \begin{itemize}
  \item \textit{Introduction à la discrétisation des EDP} :
    1\textsuperscript{ère}~année~ENSTA, chargé de TD (resp: P. Joly) ;
  \item \textit{La méthode des éléments finis} :
    2\textsuperscript{ème}~année~ENSTA, chargé de TD (resp: S. Fliss) ;
  \item \textit{Modélisation et simulation du transport de particules neutres} :
    Master Analyse, Modélisation et \mbox{Simulation} (AMS), co-responsable avec
    G. Samba (CEA);
  \item Membre du conseil de perfectionnement de l'université Paris Saclay.
  \end{itemize}
}

\section{Compétences et sujets de recherche}

\subsection{Communication et animation d'activités scientifiques}

\cvitem{Encadrement}{Suivi de quatre thèses avec divers partenaires : \small École Polytechnique
  de Montréal, Imperial College, Université Pierre et Marie Curie, CEA ;}

\cvitem{Communication scientifique}{Articles et présentations dans des revues et conférences
  internationales, \small\begin{itemize}
  \item Relecture d'articles pour la revue \textit{Nuclear Science and
      Engineering},
  \item Membre du comité de programme de la conférence \textit{Numerical
      Software Verification} (NSV 2017).
  \end{itemize}\vspace{-1em}\strut
}

% \subsection{Mathématiques Appliquées, Analyse Numérique}
% \cvitem{Méthodes num.}{Méthode des caractéristiques, éléments finis.}
% \cvitem{Calcul}{Matlab, Maxima (calcul formel).}

% \subsection{Développement de solveurs à haut rendement}

% \cvitem{Langages}{C++, Python.}
% \cvitem{Paradigmes}{Méta-programmation, \textit{multithreading}, vectorisation.}

\subsection{Développement de solveurs à haut rendement, Mathématiques
  appliquées}
\cvitem{Méthodes num.}{Méthode des caractéristiques, éléments finis ;}
\cvitem{Langages}{[production] C++, Python,
  \hfill[mise au point] Matlab, Maxima (calcul formel) ;}
\cvitem{Paradigmes}{Méta-programmation, \textit{multithreading}, vectorisation.}

\subsection{Qualité numérique des codes}

\cvitem{Arithmétique}{Arithmétiques flottante et stochastique, détection des
  erreurs d'arrondi ;}

\cvitem{Outils d'analyse}{Analyse des instabilités numériques :\small
  \begin{itemize}
  \item Utilisation de la bibliothèque \textsc{Cadna} (LIP6) ;
  \item Développement de l'outil \verrou{} et application aux codes métier EDF (Code\_Aster, Apogene).
  \end{itemize}\vspace{-1em}\strut
}

\subsection{Physique des Réacteurs : modélisation et simulation}
\cvitem{Chaîne industrielle}{%
  Développement dans la future chaîne de calcul des c{\oe}urs \textsc{Cocagne} :\small
  \begin{itemize}
  \item Solveurs d'évolution des concentrations isotopiques,
  \item Méthodes de reconstruction fine du flux,
  \item Fonctionnalités de \textit{rod cusping} pour les calculs
    instationnaires.
  \end{itemize}\vspace{-1em}\strut
}

\cvitem{Schémas avancés}{\vspace{-0.85em}\small\begin{itemize}
  \item Mise au point de la méthode d'accélération du solveur de transport
    3D cartésien \textsc{Domino},
  \item Développement du solveur de transport 3D en géométries non structurées \textsc{Micado}.
  \end{itemize}\vspace{-1em}\strut
}

\newpage

\section{Formation}

\cventry{2005--2008}{Thèse}{Commissariat à l'Énergie Atomique}{Saclay}{}{
  Résolution numérique de l'équation du transport des neutrons avec la méthode
  des caractéristiques~:
  \begin{itemize}
  \item \textit{Best student paper award}, conférence Mathematics \& Computations
    2007 ;
  \item \textit{Best student award}, conférence IYCE 2007.
  \end{itemize}
}

\cventry{2002--2005}{Cursus ingénieur}{École Nationale Supérieure de Techniques Avancées (ENSTA)}{}{}{
  \begin{description}
  \item[Spécialité ``modélisation des systèmes physiques'' :] calcul
    scientifique et parallèle, commande des systèmes, optimisation
    différentiable, physique des réacteurs, propagation des ondes.
  \end{description}}

\cventry{2004--2005}{Master Modélisation et Simulation}{INSTN}{Saclay}{}{}


\section{Doctorants}

\cvitem{Wesley Ford\\(2016--)}{Vers des méthodes d'accélération stables et
  efficaces en contextes parallèles appliquées à l'équation du transport des
  neutrons~:\small\begin{itemize}
  \item Directeur : Christophe Calvin (CEA, Maison de la Simulation).
  \end{itemize}\vspace{-1em}\strut}

\cvitem{Romain Picot\\(2015--2018)}{Vérification numérique de logiciels de calculs
  industriels~:\small\begin{itemize}
  \item Directrice : Fabienne Jézéquel (Université Pierre et Marie Curie, Paris 6).
  \end{itemize}\vspace{-1em}\strut}

\cvitem{Rebecca Jeffers\\(2013--2016)}{Goal Based Coupled Adaptive Mesh
  Refinement (AMR) and angular adaptivity on Cartesian Meshes for Modelling
  Neutron Transport in PWR Reactor Cores~:\small\begin{itemize}
  \item Directeur : Matthew D. Eaton (Imperial College London).
  \end{itemize}\vspace{-1em}\strut}

\cvitem{\hspace{-5em}Marc-André Lajoie\\(2011--2017)}{Développement d’un schéma
  de calcul prismatique généralisé parallèle en transport déterministe
  hétérogène 3-D~:\small\begin{itemize}
  \item Co-directeur : Guy Marleau (École Polytechnique de Montréal).
  \end{itemize}\vspace{-1em}\strut}

\section{Publications}
\documentclass{moderncv}
\usepackage[a4paper,hmargin=1.5cm,vmargin={1.5cm, 2cm}]{geometry}
\usepackage[utf8]{inputenc}
\moderncvstyle{ff}
\moderncvcolor{blue}
\usepackage{totcount}
\usepackage{enumitem}

\name{}{}
\title{}

\begin{document}
\makecvtitle


\input{bibtype-\bibname}

\bibliographystyle{moderncv}
\newtotcounter{NbRefs}
\newcounter{BibIndex}
\setcounter{BibIndex}{\totvalue{NbRefs}}
\newcommand{\bibindex}{\theBibIndex\addtocounter{BibIndex}{-1}\stepcounter{NbRefs}}
\bibliography{../publications/biblio}

\end{document}

\bibliographystyle{moderncv}
\newtotcounter{NbRefs}
\newcounter{BibIndex}
\setcounter{BibIndex}{\totvalue{NbRefs}}
\newcommand{\bibindex}{\theBibIndex\addtocounter{BibIndex}{-1}\stepcounter{NbRefs}}
\bibliography{../publications/biblio}

\end{document}
